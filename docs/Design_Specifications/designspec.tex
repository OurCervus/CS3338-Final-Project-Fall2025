\documentclass{article}

\usepackage{geometry}
\usepackage{indentfirst}
\usepackage{fancyhdr}
\usepackage{hyperref}
\usepackage{tabularx}

\pagestyle{fancy}
\geometry{a4paper, margin=1in}
\fancyhf{}
\fancyhead[C]{Design Specification}
\fancyfoot[C]{\thepage}

\begin{document}

\title{Design Specification \\ \large Role-Based Internal Platform for Targeted Content Delivery}
\author{Group 20}
\date{December 5, 2025}

\maketitle
\newpage
\tableofcontents
\newpage

\section{Introduction}

This Design Specification outlines the structural breakdown of the Platform, including:
\begin{itemize}
    \item All frontend pages.
    \item All backend modules.
    \item Tools and technologies used.
    \item A complete list of packages, dependencies, and Docker resources required.
\end{itemize}

It supports both the SRS and SDD by providing a concrete implementation plan.

\section{Frontend Design}

The frontend is a React-based web interface.

\subsection{Login Page}
\begin{itemize}
    \item Fields: Email, Password.
    \item Actions: Log In.
    \item Behavior:
    \begin{itemize}
        \item Sends POST /api/auth/login.
        \item Redirects based on user role.
    \end{itemize}
\end{itemize}

\subsection{End User Dashboard}
\begin{itemize}
    \item Displays personalized targeted content (announcements, policies, training).
    \item Filters: Category, Status (All, Mandatory, Completed).
    \item Search bar for titles/keywords.
    \item Each content card shows title, summary, mandatory badge.
\end{itemize}

\subsection{Content Detail Page}
\begin{itemize}
    \item Full content body, metadata, start/end dates.
    \item Acknowledge button for mandatory content.
    \item Sends:
    \begin{itemize}
        \item POST /api/content/\{id\}/view
        \item POST /api/content/\{id\}/acknowledge
    \end{itemize}
\end{itemize}

\subsection{Admin Content List}
\begin{itemize}
    \item Table of all content items.
    \item Actions: Create, Edit, Archive.
    \item Columns: Title, Mandatory, Dates, Status.
\end{itemize}

\subsection{Content Editor \& Targeting Rules Page}
\begin{itemize}
    \item Fields: Title, Body, Category, Start/End Dates, Mandatory Flag.
    \item Targeting Rules:
    \begin{itemize}
        \item Field (role, department, location)
        \item Operator
        \item Value
    \end{itemize}
\end{itemize}

\subsection{Reporting Dashboard}
\begin{itemize}
    \item Displays views, acknowledgements, completion rates.
    \item Export CSV option.
\end{itemize}

\subsection{User Management Page}
\begin{itemize}
    \item Create/Update users.
    \item Assign roles.
    \item Activate/Deactivate accounts.
\end{itemize}

\section{Backend Design}

The backend is one Python app that the website talks to. It handles things like logging in, checking rules, and getting data from the database.

\subsection{Auth Module}
\begin{itemize}
    \item Login, logout, password validation.
    \item JWT-based session management.
    \item Role-based access checks.
\end{itemize}

\subsection{User \& Role Management}
\begin{itemize}
    \item CRUD operations for users.
    \item Role assignments.
\end{itemize}

\subsection{Content Management}
\begin{itemize}
    \item Create, edit, schedule, archive content.
    \item Store metadata for targeting.
\end{itemize}

\subsection{Targeting Engine}
\begin{itemize}
    \item Evaluates targeting rules against user attributes.
    \item Supports AND/OR logic groups.
\end{itemize}

\subsection{Delivery Service}
\begin{itemize}
    \item Generates personalized content feed.
    \item Logs view and acknowledgement events.
\end{itemize}

\subsection{Reporting Engine}
\begin{itemize}
    \item Aggregates engagement data.
    \item Returns compliance summaries.
\end{itemize}

\section{Database Design}

Core tables used:
\begin{itemize}
    \item \textbf{User}: id, email, password hash, department, location, active flag.
    \item \textbf{Role}: id, name.
    \item \textbf{UserRole}: user\_id, role\_id.
    \item \textbf{ContentItem}: id, title, body, mandatory, dates.
    \item \textbf{TargetRule}: id, content\_id, field, operator, value.
    \item \textbf{DeliveryEvent}: id, content\_id, user\_id, type, timestamp.
\end{itemize}

\section{Tools Used}

\subsection{Languages \& Frameworks}
\begin{itemize}
    \item Python (backend)
    \item Flask (API framework)
    \item React (frontend)
    \item JavaScript/TypeScript
    \item PostgreSQL (database)
\end{itemize}

\subsection{Development Tools}
\begin{itemize}
    \item Docker
    \item Jira 
    \item GitHub
    \item LaTeX
\end{itemize}

\section{Dependencies \& Packages}

\subsection{Backend Dependencies (Python)}
\begin{itemize}
    \item Flask
    \item Flask-JWT-Extended (authentication)
    \item SQLAlchemy (ORM)
    \item psycopg2-binary (PostgreSQL connector)
    \item passlib[bcrypt] (password hashing)
    \item python-dotenv
\end{itemize}

\subsection{Frontend Dependencies (React)}
\begin{itemize}
    \item react
    \item react-dom
    \item react-router-dom
    \item axios (API calls)
    \item UI library (Material UI or similar)
\end{itemize}

\subsection{Dev Dependencies}
\begin{itemize}
    \item eslint, prettier
    \item typescript (if used)
\end{itemize}

\subsection{Docker Components}
\begin{itemize}
    \item Backend Container:
          \begin{itemize}
              \item Base: python:3.12-slim
              \item Installs backend dependencies
          \end{itemize}
    \item Frontend Container:
          \begin{itemize}
              \item Base: node:20-alpine
              \item Builds React app
          \end{itemize}
    \item Database Container:
          \begin{itemize}
              \item Base: postgres:16-alpine
              \item Stores persistent data
          \end{itemize}
\end{itemize}


\end{document}
