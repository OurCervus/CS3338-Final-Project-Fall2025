\documentclass{article} 

\usepackage{geometry}
\usepackage{indentfirst}
\usepackage{fancyhdr}
\usepackage{caption}
\usepackage{graphicx}
\usepackage{enumitem}
\usepackage[
    colorlinks=true,
    linkcolor=black,
    urlcolor=blue,
    citecolor=black
]{hyperref}
\usepackage{tabularx}

\pagestyle{fancy}
\geometry{a4paper, margin=1in}
\captionsetup{labelformat=empty}
\graphicspath{ {../../assets/} }

\begin{document}

\title{Software Design Document \\ \large Role-Based Internal Platform for Targeted Content Delivery}
\author{Version 1.0.0}
\date{
       Group 20
   \\ Jesus Calvo,
    \\ Justin Gip
    \\ Shane Guzman
    \\ Samuel Mekonnen
    \\ ------------------
    \\ December 5, 2025
    \\ CS 3338-01 (94305)
    }

\maketitle
\newpage
\tableofcontents
\newpage

\fancyhf{}
\fancyhead[C]{Software Design Document}
\fancyfoot[C]{\thepage}

\section*{Version Description}
\begin{table}[h!]
\centering
\caption{\textbf{Version Description}}
\begin{tabularx}{\textwidth}{|l|l|X|l|}
\hline
\textbf{Author} & \textbf{Date} & \textbf{Description} & \textbf{Version} \\ \hline
Your Name & 10/15/25 & Initial draft of SDD created for project proposal & 0.1.0 \\ \hline
Your Name & 11/05/25 & Updated architecture and data design for Snapshot 2 & 0.2.0 \\ \hline
Team & 12/05/25 & Finalized design after feedback and implementation changes & 1.0.0 \\ \hline
\end{tabularx}
\end{table}

\section{Introduction}

\subsection{Purpose}
This Software Design Document (SDD) describes the technical level and detailed design of the \textbf{Role-Based Internal Platform for Targeted Content Delivery} (the Platform). 
It translates the functional and nonfunctional requirements defined in the Software Requirements Specification (SRS) into a architecture, technologies, and modules.
The SDD is intended to guide developers, testers, and operations staff as they implement, test, deploy, and maintain the system.

\subsection{Scope}
The Platform is an internal web app used by organizations to deliver targeted announcements, training materials, and resources to employees based on their role, department, location, and other attributes.
It includes:
\begin{itemize}
    \item A responsive web user interface for end users and administrators.
    \item Role based authentication and authorization for all internal users.
    \item A content management module for creating, tagging, and scheduling content.
    \item A targeting rules engine that matches content to audiences.
    \item Delivery channels for showing content in dashboards and sending notification by email.
    \item Analytics and reporting for content engagement and compliance.
\end{itemize}

\subsection{Intended Audience}
\begin{itemize}
    \item \textbf{Developers} use this document to understand the system architecture, module responsibilities, and interfaces.
    \item \textbf{Testers} use the detailed design to derive test plans, cases, and data needed to verify behavior.
    \item \textbf{Project managers} use the SDD to understand the scope and complexity of the solution.
    \item \textbf{Operations / DevOps} use the deployment view and nonfunctional design aspects.
\end{itemize}

\subsection{References}
\begin{itemize}
    \item Software Requirements Specification (SRS) for Role-Based Internal Platform for Targeted Content Delivery.
    \item Jira space
\end{itemize}

\subsection{Definitions, Acronyms, and Abbreviations}
\begin{itemize}
    \item \textbf{RBAC} Role Based Access Control.
    \item \textbf{CMS} Content Management System.
    \item \textbf{NFR} Nonfunctional Requirement.
    \item \textbf{SRS} Software Requirements Specification.
    \item \textbf{SDD} Software Design Document.
\end{itemize}

\begin{table}[h!]
\centering
\caption{\textbf{Glossary of Acronyms}}
\begin{tabularx}{0.8\textwidth}{|l|X|}
\hline
\textbf{Acronym} & \textbf{Definition} \\ \hline
RBAC & Role Based Access Control \\ \hline
CMS & Content Management System \\ \hline
NFR & Nonfunctional Requirement \\ \hline
SRS & Software Requirements Specification \\ \hline
SDD & Software Design Document \\ \hline
\end{tabularx}
\end{table}

\section{System Architecture}

\subsection{Architectural Goals and Constraints}
The system architecture is designed with the following goals:
\begin{itemize}
    \item Enforce strong role based access control for all content and actions.
    \item Support flexible content targeting rules without code changes.
    \item Be deployable in a containerized environment (e.g., Docker).
    \item Be maintainable for a small team.
\end{itemize}

Constraints:
\begin{itemize}
    \item Web based user interface built using technologies covered in class (e.g., Jira, Docker, LaTex).
    \item Back end implemented as a single service to keep the project manageable.
    \item Uses a relational database such as PostgreSQL.
\end{itemize}

\subsection{High Level Architecture}
At a high level, the Platform follows a layered architecture with the following layers:
\begin{itemize}
    \item \textbf{Presentation Layer} Web UI (end user portal and admin portal).
    \item \textbf{Application Layer} HTTP API, authentication, content management, targeting engine, analytics.
    \item \textbf{Data Layer} Relational database and persistence logic.
    \item \textbf{Integration Layer} Email service or notification adapter.
\end{itemize}
\begin{itemize}
    \item Users interacting with the Web UI via browser.
    \item The Web UI making requests to the Application Layer (REST API).
    \item The Application Layer reading and writing data in the Data Layer.
    \item Optional outgoing calls to an email or notification service.
\end{itemize}

\section{Component Design}

\subsection{Component Overview}
The main components of the Platform are:
\begin{itemize}
    \item \textbf{Auth Service} Handles login, logout, password reset, and enforcement of RBAC.
    \item \textbf{User and Role Management} Stores users, roles, permissions, and organizational attributes.
    \item \textbf{Content Management} Allows admins to create, edit, schedule, and retire content items.
    \item \textbf{Targeting Engine} Evaluates which users should see which content given a set of rules.
    \item \textbf{Delivery Service} Exposes content to the UI and optionally triggers outbound emails.
    \item \textbf{Analytics and Reporting} Aggregates view and click events and exposes metrics.
\end{itemize}

\subsection{Auth Service}
\subsubsection{Responsibilities}
\begin{itemize}
    \item Authenticate users via email and password.
    \item Issue and validate session tokens or cookies.
    \item Attach user identity and role information to each request.
\end{itemize}

\subsubsection{Interfaces}
\begin{itemize}
    \item POST /api/auth/login
    \item POST /api/auth/logout
    \item GET /api/auth/me
\end{itemize}

\subsection{User and Role Management}
\subsubsection{Responsibilities}
\begin{itemize}
    \item Create, read, update, and deactivate user accounts.
    \item Maintain mapping of users to roles (e.g., Employee, Manager, HR, Admin).
    \item Maintain permissions per role for actions such as content creation, publishing, and reporting.
\end{itemize}

\subsubsection{Interfaces}
\begin{itemize}
    \item GET /api/users
    \item POST /api/users
    \item GET /api/roles
    \item POST /api/roles
\end{itemize}

\subsection{Content Management}
\subsubsection{Responsibilities}
\begin{itemize}
    \item Create and edit content items (title, body, attachments, tags).
    \item Schedule content start and end dates.
    \item Assign targeting rules to content.
\end{itemize}

\subsubsection{Interfaces}
\begin{itemize}
    \item GET /api/content
    \item POST /api/content
    \item PUT /api/content/\{id\}
    \item DELETE /api/content/\{id\}
\end{itemize}

\subsection{Targeting Engine}
\subsubsection{Responsibilities}
\begin{itemize}
    \item Evaluate target audiences based on user attributes (role, department, location) and rule definitions.
    \item Cache evaluations where possible to reduce database load.
\end{itemize}

\subsubsection{Rule Model}
Targeting rules are represented as a set of conditions:
\begin{itemize}
    \item Field (e.g., role, department, location).
    \item Operator (equals, in list, not equals).
    \item Value(s) (e.g., ``Manager'', ``HR'', ``US'').
\end{itemize}
Rules can be combined using logical AND and OR groups.

\subsection{Delivery Service}
\subsubsection{Responsibilities}
\begin{itemize}
    \item Provide the current user with the list of active content items they should see.
    \item Log each view or click as an event.
    \item Optionally send notification emails when new content is published.
\end{itemize}

\subsubsection{Interfaces}
\begin{itemize}
    \item GET /api/feed -- returns targeted content for the logged in user.
    \item POST /api/events -- records content view or click events.
\end{itemize}

\subsection{Analytics and Reporting}
\subsubsection{Responsibilities}
\begin{itemize}
    \item Aggregate events by content item, audience, and time window.
    \item Expose metrics such as views, unique viewers, and completion rate for required content.
\end{itemize}

\subsubsection{Interfaces}
\begin{itemize}
    \item GET /api/reports/overview
    \item GET /api/reports/content/\{id\}
\end{itemize}

\section{Data Design}

\subsection{Data Model Overview}
The Platform uses a relational database with normalized tables.
Key entities include:
\begin{itemize}
    \item User
    \item Role
    \item UserRole (join table)
    \item ContentItem
    \item TargetRule
    \item DeliveryEvent
\end{itemize}

\subsection{Core Tables}

\subsubsection{User}
\begin{itemize}
    \item id (primary key)
    \item email (unique)
    \item password\_hash
    \item first\_name
    \item last\_name
    \item department
    \item location
    \item is\_active
    \item created\_at
\end{itemize}

\subsubsection{Role}
\begin{itemize}
    \item id (primary key)
    \item name (e.g., Employee, Manager, HR, Admin)
    \item description
\end{itemize}

\subsubsection{ContentItem}
\begin{itemize}
    \item id (primary key)
    \item title
    \item body
    \item category
    \item is\_mandatory (boolean)
    \item start\_date
    \item end\_date
    \item created\_by\_user\_id (foreign key to User)
\end{itemize}

\subsubsection{TargetRule}
\begin{itemize}
    \item id (primary key)
    \item content\_id (foreign key to ContentItem)
    \item field (e.g., ``role'', ``department'', ``location'')
    \item operator (e.g., ``EQUALS'', ``IN'')
    \item value
\end{itemize}

\subsubsection{DeliveryEvent}
\begin{itemize}
    \item id (primary key)
    \item content\_id (foreign key)
    \item user\_id (foreign key)
    \item event\_type (view, click, dismiss)
    \item occurred\_at (timestamp)
\end{itemize}

\section{User Interface Design}

\subsection{Screen Overview}
The main screens are:
\begin{itemize}
    \item \textbf{Login Screen} Authenticates the user into the Platform.
    \item \textbf{End User Dashboard} Shows a personalized feed of targeted content items, including filters and search.
    \item \textbf{Content Detail Page} Displays full content, attachments, and required acknowledgement actions.
    \item \textbf{Admin Content List} Lists all content items with status, author, and filters.
    \item \textbf{Content Editor} Form for creating and editing content and targeting rules.
    \item \textbf{Reports Dashboard} Displays analytics charts and tables.
\end{itemize}

\subsection{Navigation Flow}
Typical navigation flow:
\begin{enumerate}
    \item User navigates to the login page and enters credentials.
    \item After successful authentication, the Platform redirects to the appropriate dashboard:
    \begin{itemize}
        \item Regular employees see the End User Dashboard.
        \item Admins see the Admin Content List or Reports Dashboard.
    \end{itemize}
    \item From the dashboard, users can click into a content item, acknowledge required content, or mark optional content as read.
    \item Admin users can navigate to content creation, targeting configuration, and reporting views.
\end{enumerate}

\section{Requirements Traceability}

 Table \ref{tab:traceability} demonstrates the requirements and its design

\begin{table}[h!]
\centering
\caption{\textbf{Requirements Traceability}}
\label{tab:traceability}
\begin{tabularx}{\textwidth}{|l|X|X|}
\hline
\textbf{Requirement ID} & \textbf{Requirement (from SRS)} & \textbf{Design Element} \\ \hline
FR-01 & The system shall authenticate users using organization credentials. & Auth Service, Login Screen \\ \hline
FR-02 & The system shall allow admins to create and publish content items. & Content Management, Admin Content List, Content Editor \\ \hline
FR-03 & The system shall show each user only the content targeted to them. & Targeting Engine, Delivery Service, End User Dashboard \\ \hline
FR-04 & The system shall track when users view or acknowledge content. & DeliveryEvent table, Delivery Service, Reports Dashboard \\ \hline
\end{tabularx}
\end{table}

\section{Nonfunctional Design Considerations}

\subsection{Performance}
\begin{itemize}
    \item Use indexed columns on foreign keys and frequently filtered fields.
    \item Cache the list of targeted content for a user within a short time window to reduce rule evaluations.
\end{itemize}

\subsection{Security}
\begin{itemize}
    \item All authenticated endpoints require a valid session or token.
    \item Sensitive data, such as password hashes, is never logged or exposed through APIs.
    \item RBAC checks are enforced in a single middleware or service to avoid duplication.
\end{itemize}

\subsection{Maintainability}
\begin{itemize}
    \item Follow consistent coding standards and naming conventions.
    \item Organize the code base by feature or layer.
    \item Write unit tests for core business logic, especially the targeting engine.
\end{itemize}

\section{Appendix A: Future Enhancements}
Potential future enhancements include:
\begin{itemize}
    \item Mobile push notifications.
    \item A self service rule builder with a graphical interface.
    \item Integration with learning management systems for training content.
\end{itemize}

\end{document}
