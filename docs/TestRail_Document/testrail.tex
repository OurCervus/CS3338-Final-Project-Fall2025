\documentclass{article}

\usepackage{geometry}
\usepackage{indentfirst}
\usepackage{fancyhdr}
\usepackage{caption}
\usepackage{enumitem}
\usepackage[
    colorlinks=true,
    linkcolor=black,
    urlcolor=blue,
    citecolor=black
]{hyperref}
\usepackage{tabularx}

\pagestyle{fancy}
\geometry{a4paper, margin=1in}
\captionsetup{labelformat=empty}

\begin{document}

\title{Test Cases and Hypothetical Test Reports \\ \large Role-Based Internal Platform for Targeted Content Delivery}
\author{Version 1.0.0}
\date{
       Group 20
   \\ Jesus Calvo,
    \\ Justin Gip
    \\ Shane Guzman
    \\ Samuel Mekonnen
    \\ ------------------
    \\ December 5, 2025
    \\ CS 3338-01 (94305)
}

\maketitle
\newpage
\tableofcontents
\newpage

\fancyhf{}
\fancyhead[C]{Test Cases and Reports}
\fancyfoot[C]{\thepage}

\section{Introduction}

This document describes the functional test cases and test execution reports for the \textbf{Role-Based Internal Platform for Targeted Content Delivery}.  
Test cases are derived from the Software Requirements Specification (SRS) and Software Design Document (SDD) and grouped by project snapshot.

Each test case includes:
\begin{itemize}
    \item A unique Test Case ID.
    \item Related functional requirement(s) (FR-\textit{nn}).
    \item Preconditions.
    \item Test steps.
    \item Expected results.
\end{itemize}

For each snapshot a short, test report is provided, summarizing the imagined outcome of executing these tests and how they relate to the implemented Jira work items.

\section{Test Strategy Overview}

\subsection{Scope}

The test cases in this document focus on validating key behaviors of:
\begin{itemize}
    \item User authentication and authorization.
    \item Content management and targeting.
    \item Targeted content feed and acknowledgements.
    \item Reporting and administration features.
\end{itemize}

Nonfunctional requirements (performance, security, usability) are touched by a small number of high-level tests.

\subsection{Test Levels}

\begin{itemize}
    \item \textbf{Unit tests} for core business logic (e.g., targeting engine, acknowledgement rules).
    \item \textbf{API-level integration tests} for endpoints like \texttt{/api/auth/login}, \texttt{/api/content}, \texttt{/api/feed}, and \texttt{/api/events}.
    \item \textbf{UI/system tests} for end-to-end flows executed via the web interface.
\end{itemize}

\subsection{Test Management}

In a real deployment, tests would be tracked using:
\begin{itemize}
    \item Jira issues for each feature and defect.
    \item A testing section or board within the Jira project for organizing manual and automated test runs.
\end{itemize}

For the purpose of this course project, this document serves as the primary record of test cases and reports.

\section{Snapshot 1 Test Cases: Authentication and Basic Navigation}

\subsection{Scope}

Snapshot 1 focuses on:
\begin{itemize}
    \item User login and basic RBAC (FR-01--FR-05).
    \item Minimal dashboard after login.
\end{itemize}

\subsection{Test Cases}

\subsubsection*{TC-1.1: Login with Valid Credentials}

\begin{itemize}[leftmargin=1.5cm]
    \item \textbf{Related Requirements:} FR-01, FR-02, FR-03
    \item \textbf{Preconditions:}
    \begin{itemize}
        \item User \texttt{employee1@example.com} exists with a known password.
        \item User account is active.
    \end{itemize}
    \item \textbf{Steps:}
    \begin{enumerate}
        \item Navigate to the Login page.
        \item Enter valid email and password for \texttt{employee1@example.com}.
        \item Click the \textbf{Login} button.
    \end{enumerate}
    \item \textbf{Expected Result:}
    \begin{itemize}
        \item User is redirected to the dashboard page.
        \item A valid session or token is created.
        \item Protected pages can now be accessed without being redirected back to login.
    \end{itemize}
\end{itemize}

\subsubsection*{TC-1.2: Login with Invalid Password}

\begin{itemize}[leftmargin=1.5cm]
    \item \textbf{Related Requirements:} FR-01, FR-02
    \item \textbf{Preconditions:}
    \begin{itemize}
        \item User \texttt{employee1@example.com} exists.
    \end{itemize}
    \item \textbf{Steps:}
    \begin{enumerate}
        \item Navigate to the Login page.
        \item Enter correct email but an incorrect password.
        \item Click the \textbf{Login} button.
    \end{enumerate}
    \item \textbf{Expected Result:}
    \begin{itemize}
        \item Login attempt is rejected.
        \item An error message such as ``Invalid email or password'' is shown.
        \item No session or token is created.
    \end{itemize}
\end{itemize}

\subsubsection*{TC-1.3: Access Protected Page When Not Authenticated}

\begin{itemize}[leftmargin=1.5cm]
    \item \textbf{Related Requirements:} FR-03
    \item \textbf{Preconditions:}
    \begin{itemize}
        \item No active session or token.
    \end{itemize}
    \item \textbf{Steps:}
    \begin{enumerate}
        \item Attempt to directly open the dashboard URL in the browser.
    \end{enumerate}
    \item \textbf{Expected Result:}
    \begin{itemize}
        \item User is redirected to the login page OR receives a \texttt{401 Unauthorized} response.
        \item No dashboard content is displayed.
    \end{itemize}
\end{itemize}

\subsubsection*{TC-1.4: Role-Based Access to Admin Console}

\begin{itemize}[leftmargin=1.5cm]
    \item \textbf{Related Requirements:} FR-04, FR-05
    \item \textbf{Preconditions:}
    \begin{itemize}
        \item \texttt{employee1} has the \texttt{Employee} role only.
        \item \texttt{admin1} has the \texttt{Content Administrator} role.
    \end{itemize}
    \item \textbf{Steps:}
    \begin{enumerate}
        \item Log in as \texttt{employee1}. Attempt to open Admin Console.
        \item Log out, then log in as \texttt{admin1}. Open Admin Console.
    \end{enumerate}
    \item \textbf{Expected Result:}
    \begin{itemize}
        \item \texttt{employee1} is denied access (error message or redirect).
        \item \texttt{admin1} can access Admin Console successfully.
    \end{itemize}
\end{itemize}

\subsection{Hypothetical Snapshot 1 Test Report}

\begin{table}[h!]
\centering
\caption{\textbf{Snapshot 1 Test Execution Summary (Hypothetical)}}
\begin{tabularx}{0.9\textwidth}{|l|c|c|c|c|}
\hline
\textbf{Test Case Group} & \textbf{\# Cases} & \textbf{Passed} & \textbf{Failed} & \textbf{Blocked} \\ \hline
Authentication (TC-1.1--TC-1.3) & 3 & 3 & 0 & 0 \\ \hline
Role-Based Access (TC-1.4) & 1 & 1 & 0 & 0 \\ \hline
\textbf{Total} & \textbf{4} & \textbf{4} & \textbf{0} & \textbf{0} \\ \hline
\end{tabularx}
\end{table}

\noindent
\textbf{Summary (Hypothetical):} All Snapshot 1 tests passed. Basic authentication and RBAC are functioning as expected, providing a stable foundation for later snapshots.

\newpage
\section{Snapshot 2 Test Cases: Content Management and Targeting}

\subsection{Scope}

Snapshot 2 adds content management and initial targeting logic:
\begin{itemize}
    \item Content creation, scheduling, and archiving (FR-06--FR-10).
    \item Attribute-based targeting rules (FR-11--FR-14).
\end{itemize}

\subsection{Test Cases}

\subsubsection*{TC-2.1: Create Content Item with Valid Data}

\begin{itemize}[leftmargin=1.5cm]
    \item \textbf{Related Requirements:} FR-06
    \item \textbf{Preconditions:}
    \begin{itemize}
        \item Logged in as \texttt{Content Administrator}.
    \end{itemize}
    \item \textbf{Steps:}
    \begin{enumerate}
        \item Navigate to \textbf{Create Content} page.
        \item Enter title, body, and select category and tags.
        \item Click \textbf{Save}.
    \end{enumerate}
    \item \textbf{Expected Result:}
    \begin{itemize}
        \item New content item is persisted and appears in Admin Content List.
        \item No validation errors are displayed.
    \end{itemize}
\end{itemize}

\subsubsection*{TC-2.2: Reject Content Creation with Missing Title}

\begin{itemize}[leftmargin=1.5cm]
    \item \textbf{Related Requirements:} FR-06
    \item \textbf{Preconditions:} Logged in as \texttt{Content Administrator}.
    \item \textbf{Steps:}
    \begin{enumerate}
        \item Open \textbf{Create Content}.
        \item Leave title empty; fill in body and category.
        \item Click \textbf{Save}.
    \end{enumerate}
    \item \textbf{Expected Result:}
    \begin{itemize}
        \item Content is not saved.
        \item Validation error indicates title is required.
    \end{itemize}
\end{itemize}

\subsubsection*{TC-2.3: Schedule Content Using Start and End Dates}

\begin{itemize}[leftmargin=1.5cm]
    \item \textbf{Related Requirements:} FR-07, FR-19
    \item \textbf{Preconditions:}
    \begin{itemize}
        \item Logged in as \texttt{Content Administrator}.
        \item Current date is between two known dates.
    \end{itemize}
    \item \textbf{Steps:}
    \begin{enumerate}
        \item Create content C1 with start date = today and end date = today + 7 days.
        \item Create content C2 with end date = yesterday.
        \item Log in as \texttt{employee1} and view feed.
    \end{enumerate}
    \item \textbf{Expected Result:}
    \begin{itemize}
        \item C1 is visible in the feed.
        \item C2 is not visible because it has expired.
    \end{itemize}
\end{itemize}

\subsubsection*{TC-2.4: Archive Content Item}

\begin{itemize}[leftmargin=1.5cm]
    \item \textbf{Related Requirements:} FR-10
    \item \textbf{Preconditions:}
    \begin{itemize}
        \item An active content item C3 exists.
    \end{itemize}
    \item \textbf{Steps:}
    \begin{enumerate}
        \item As \texttt{Content Administrator}, open C3 in the Admin Content List.
        \item Click \textbf{Archive} and confirm.
        \item Log in as targeted end user and view feed.
    \end{enumerate}
    \item \textbf{Expected Result:}
    \begin{itemize}
        \item C3 is no longer shown to end users.
        \item C3 appears with status \texttt{Archived} in Admin Content List.
    \end{itemize}
\end{itemize}

\subsubsection*{TC-2.5: Simple Role-Based Targeting Rule}

\begin{itemize}[leftmargin=1.5cm]
    \item \textbf{Related Requirements:} FR-11, FR-13
    \item \textbf{Preconditions:}
    \begin{itemize}
        \item User \texttt{employee1} with role = Employee.
        \item User \texttt{manager1} with role = Manager.
    \end{itemize}
    \item \textbf{Steps:}
    \begin{enumerate}
        \item Create content C4 targeted to role = Manager.
        \item Log in as \texttt{employee1} and view feed.
        \item Log in as \texttt{manager1} and view feed.
    \end{enumerate}
    \item \textbf{Expected Result:}
    \begin{itemize}
        \item \texttt{employee1} does not see C4.
        \item \texttt{manager1} sees C4 in the feed.
    \end{itemize}
\end{itemize}

\subsubsection*{TC-2.6: Update Targeting After Role Change}

\begin{itemize}[leftmargin=1.5cm]
    \item \textbf{Related Requirements:} FR-14
    \item \textbf{Preconditions:}
    \begin{itemize}
        \item Content C5 targeted to role = Manager.
        \item User \texttt{employee2} currently role = Employee.
    \end{itemize}
    \item \textbf{Steps:}
    \begin{enumerate}
        \item Log in as \texttt{employee2}; verify C5 is not visible.
        \item Change \texttt{employee2}'s role to Manager using User Administration.
        \item Log in again as \texttt{employee2} and view feed.
    \end{enumerate}
    \item \textbf{Expected Result:}
    \begin{itemize}
        \item Initially C5 is not in the feed.
        \item After role change, C5 appears for \texttt{employee2}.
    \end{itemize}
\end{itemize}

\subsection{Hypothetical Snapshot 2 Test Report}

\begin{table}[h!]
\centering
\caption{\textbf{Snapshot 2 Test Execution Summary (Hypothetical)}}
\begin{tabularx}{0.9\textwidth}{|l|c|c|c|c|}
\hline
\textbf{Test Case Group} & \textbf{\# Cases} & \textbf{Passed} & \textbf{Failed} & \textbf{Blocked} \\ \hline
Content Creation and Validation (TC-2.1--TC-2.2) & 2 & 2 & 0 & 0 \\ \hline
Scheduling and Archiving (TC-2.3--TC-2.4) & 2 & 1 & 1 & 0 \\ \hline
Targeting Rules (TC-2.5--TC-2.6) & 2 & 2 & 0 & 0 \\ \hline
\textbf{Total} & \textbf{6} & \textbf{5} & \textbf{1} & \textbf{0} \\ \hline
\end{tabularx}
\end{table}

\noindent
\textbf{Summary (Hypothetical):} One failure occurred in TC-2.3, where expired content still appeared in the feed due to an off-by-one comparison bug on end dates. A Jira bug ticket was created and fixed in Snapshot 3.

\newpage
\section{Snapshot 3 Test Cases: Targeted Feed, Acknowledgements, and Reporting}

\subsection{Scope}

Snapshot 3 delivers:
\begin{itemize}
    \item Targeted content feed (FR-15--FR-19).
    \item View and acknowledgement tracking (FR-20--FR-23).
    \item Initial reporting (FR-24--FR-26).
\end{itemize}

\subsection{Test Cases}

\subsubsection*{TC-3.1: Personalized Feed Shows Only Targeted Content}

\begin{itemize}[leftmargin=1.5cm]
    \item \textbf{Related Requirements:} FR-15, FR-19
    \item \textbf{Preconditions:}
    \begin{itemize}
        \item Content C6 targeted to role = Employee.
        \item Content C7 targeted to role = Manager.
        \item User \texttt{employee1} has role = Employee.
    \end{itemize}
    \item \textbf{Steps:}
    \begin{enumerate}
        \item Log in as \texttt{employee1}.
        \item Open End User Dashboard.
    \end{enumerate}
    \item \textbf{Expected Result:}
    \begin{itemize}
        \item C6 appears in feed.
        \item C7 does not appear in feed.
    \end{itemize}
\end{itemize}

\subsubsection*{TC-3.2: Feed Filtering by Category}

\begin{itemize}[leftmargin=1.5cm]
    \item \textbf{Related Requirements:} FR-17
    \item \textbf{Preconditions:}
    \begin{itemize}
        \item Two visible content items exist: C8 (category = Compliance) and C9 (category = General).
    \end{itemize}
    \item \textbf{Steps:}
    \begin{enumerate}
        \item Log in and open dashboard.
        \item Apply filter \textbf{Category = Compliance}.
    \end{enumerate}
    \item \textbf{Expected Result:}
    \begin{itemize}
        \item C8 remains visible.
        \item C9 is hidden from the list.
    \end{itemize}
\end{itemize}

\subsubsection*{TC-3.3: Search Content by Keyword}

\begin{itemize}[leftmargin=1.5cm]
    \item \textbf{Related Requirements:} FR-18
    \item \textbf{Steps:}
    \begin{enumerate}
        \item Ensure there is content C10 with the title ``Privacy Policy Update''.
        \item Enter keyword ``Privacy'' into search box.
    \end{enumerate}
    \item \textbf{Expected Result:}
    \begin{itemize}
        \item C10 appears in search results.
        \item Unrelated content is excluded from results.
    \end{itemize}
\end{itemize}

\subsubsection*{TC-3.4: Record View Event when Content is Opened}

\begin{itemize}[leftmargin=1.5cm]
    \item \textbf{Related Requirements:} FR-20
    \item \textbf{Steps:}
    \begin{enumerate}
        \item Log in as \texttt{employee1}.
        \item Open detail page for content C11.
        \item Query \texttt{DeliveryEvent} table or reporting API for events for C11 and \texttt{employee1}.
    \end{enumerate}
    \item \textbf{Expected Result:}
    \begin{itemize}
        \item A \texttt{view} event exists for C11 and \texttt{employee1} with a recent timestamp.
    \end{itemize}
\end{itemize}

\subsubsection*{TC-3.5: Acknowledge Mandatory Content}

\begin{itemize}[leftmargin=1.5cm]
    \item \textbf{Related Requirements:} FR-21, FR-22, FR-23
    \item \textbf{Preconditions:} C12 is marked as mandatory for \texttt{employee1}.
    \item \textbf{Steps:}
    \begin{enumerate}
        \item Log in as \texttt{employee1} and open C12.
        \item Click \textbf{Acknowledge}.
        \item Attempt to call API as a different user to acknowledge C12 on behalf of \texttt{employee1}.
    \end{enumerate}
    \item \textbf{Expected Result:}
    \begin{itemize}
        \item Acknowledgement is recorded with timestamp for \texttt{employee1}.
        \item Attempt to acknowledge on behalf of another user is rejected with an error.
    \end{itemize}
\end{itemize}

\subsubsection*{TC-3.6: Content Overview Report Shows Correct Counts}

\begin{itemize}[leftmargin=1.5cm]
    \item \textbf{Related Requirements:} FR-24
    \item \textbf{Preconditions:}
    \begin{itemize}
        \item Known numbers of targeted, viewed, and acknowledged users for content C13.
    \end{itemize}
    \item \textbf{Steps:}
    \begin{enumerate}
        \item As Content Administrator, open Reports Dashboard.
        \item Locate row for C13.
    \end{enumerate}
    \item \textbf{Expected Result:}
    \begin{itemize}
        \item Report displays counts that match the underlying \texttt{DeliveryEvent} data.
    \end{itemize}
\end{itemize}

\subsection{Hypothetical Snapshot 3 Test Report}

\begin{table}[h!]
\centering
\caption{\textbf{Snapshot 3 Test Execution Summary (Hypothetical)}}
\begin{tabularx}{0.9\textwidth}{|l|c|c|c|c|}
\hline
\textbf{Test Case Group} & \textbf{\# Cases} & \textbf{Passed} & \textbf{Failed} & \textbf{Blocked} \\ \hline
Targeted Feed (TC-3.1--TC-3.3) & 3 & 3 & 0 & 0 \\ \hline
Acknowledgements and Events (TC-3.4--TC-3.5) & 2 & 2 & 0 & 0 \\ \hline
Reporting (TC-3.6) & 1 & 1 & 0 & 0 \\ \hline
\textbf{Total} & \textbf{6} & \textbf{6} & \textbf{0} & \textbf{0} \\ \hline
\end{tabularx}
\end{table}

\noindent
\textbf{Summary (Hypothetical):} All Snapshot 3 tests passed after the date-handling bug from Snapshot 2 was fixed. Reporting metrics align with event data.

\newpage
\section{Snapshot 4 Test Cases: Administration and Regression}

\subsection{Scope}

Snapshot 4 finalizes:
\begin{itemize}
    \item User and role administration (FR-27--FR-30).
    \item Regression coverage over previous snapshots.
\end{itemize}

\subsection{Test Cases}

\subsubsection*{TC-4.1: Create New User Account}

\begin{itemize}[leftmargin=1.5cm]
    \item \textbf{Related Requirements:} FR-27, FR-28
    \item \textbf{Steps:}
    \begin{enumerate}
        \item Log in as System Administrator.
        \item Open User Management page.
        \item Create a new user with email, name, and role = Employee.
        \item Save.
    \end{enumerate}
    \item \textbf{Expected Result:}
    \begin{itemize}
        \item New user appears in user list.
        \item User can log in with initial credentials (after password setup flow, if applicable).
    \end{itemize}
\end{itemize}

\subsubsection*{TC-4.2: Deactivate and Reactivate User}

\begin{itemize}[leftmargin=1.5cm]
    \item \textbf{Related Requirements:} FR-27
    \item \textbf{Steps:}
    \begin{enumerate}
        \item Deactivate existing user \texttt{employee3}.
        \item Attempt to log in as \texttt{employee3}.
        \item Reactivate \texttt{employee3}.
        \item Attempt login again.
    \end{enumerate}
    \item \textbf{Expected Result:}
    \begin{itemize}
        \item Deactivated user cannot log in.
        \item Reactivated user can log in successfully.
    \end{itemize}
\end{itemize}

\subsubsection*{TC-4.3: Reset User Password}

\begin{itemize}[leftmargin=1.5cm]
    \item \textbf{Related Requirements:} FR-29
    \item \textbf{Steps:}
    \begin{enumerate}
        \item As System Administrator, request password reset for \texttt{employee4}.
        \item Complete reset flow (e.g., admin-set or simulated email token).
        \item Attempt login using new password.
    \end{enumerate}
    \item \textbf{Expected Result:}
    \begin{itemize}
        \item Login works only with new password.
        \item Old password no longer works.
    \end{itemize}
\end{itemize}

\subsubsection*{TC-4.4: Audit Log Entry for Administrative Change}

\begin{itemize}[leftmargin=1.5cm]
    \item \textbf{Related Requirements:} FR-30
    \item \textbf{Steps:}
    \begin{enumerate}
        \item As System Administrator, change the role of \texttt{employee4} from Employee to Manager.
        \item Open the audit log or administrative change history.
    \end{enumerate}
    \item \textbf{Expected Result:}
    \begin{itemize}
        \item Log includes an entry recording who changed which user, from which role to which role, and when.
    \end{itemize}
\end{itemize}

\subsubsection*{TC-4.5: Regression -- Authentication, Content, and Feed Smoke Test}

\begin{itemize}[leftmargin=1.5cm]
    \item \textbf{Related Requirements:} FR-01--FR-26 (high-level coverage)
    \item \textbf{Steps:}
    \begin{enumerate}
        \item Log in as employee and verify dashboard loads.
        \item Ensure at least one targeted content item appears.
        \item Open the content, verify a view event is recorded, and acknowledge if mandatory.
        \item As Admin, verify that reporting reflects the new view and acknowledgement.
    \end{enumerate}
    \item \textbf{Expected Result:}
    \begin{itemize}
        \item No regressions in core authentication, content visibility, acknowledgement, or reporting paths.
    \end{itemize}
\end{itemize}

\subsection{Hypothetical Snapshot 4 Test Report}

\begin{table}[h!]
\centering
\caption{\textbf{Snapshot 4 Test Execution Summary (Hypothetical)}}
\begin{tabularx}{0.9\textwidth}{|l|c|c|c|c|}
\hline
\textbf{Test Case Group} & \textbf{\# Cases} & \textbf{Passed} & \textbf{Failed} & \textbf{Blocked} \\ \hline
User and Role Management (TC-4.1--TC-4.4) & 4 & 4 & 0 & 0 \\ \hline
Regression Smoke Test (TC-4.5) & 1 & 1 & 0 & 0 \\ \hline
\textbf{Total} & \textbf{5} & \textbf{5} & \textbf{0} & \textbf{0} \\ \hline
\end{tabularx}
\end{table}

\noindent
\textbf{Summary (Hypothetical):} All Snapshot 4 administration and regression tests passed, indicating the system is ready for final demonstration.

\newpage
\section{Appendix A: Requirement-to-Test Traceability (Partial)}

\begin{table}[h!]
\centering
\caption{\textbf{Sample Traceability Between Requirements and Test Cases}}
\begin{tabularx}{\textwidth}{|l|X|X|}
\hline
\textbf{Requirement ID} & \textbf{Requirement Summary} & \textbf{Covering Test Case IDs} \\ \hline
FR-01 & User login with email and password & TC-1.1, TC-1.2, TC-4.5 \\ \hline
FR-03 & Prevent access when not logged in & TC-1.3 \\ \hline
FR-05 & Restrict admin pages to roles & TC-1.4 \\ \hline
FR-06 & Create content items & TC-2.1, TC-2.2 \\ \hline
FR-07 \& FR-19 & Start/end dates and expired content & TC-2.3, TC-3.1 \\ \hline
FR-11--FR-14 & Targeting rules and updates & TC-2.5, TC-2.6, TC-3.1 \\ \hline
FR-15--FR-18 & Targeted feed, filters, search & TC-3.1, TC-3.2, TC-3.3 \\ \hline
FR-20--FR-23 & Views and acknowledgements & TC-3.4, TC-3.5, TC-4.5 \\ \hline
FR-24--FR-26 & Reporting and exports & TC-3.6, TC-4.5 \\ \hline
FR-27--FR-30 & User administration and audit logs & TC-4.1--TC-4.4 \\ \hline
\end{tabularx}
\end{table}

\end{document}
